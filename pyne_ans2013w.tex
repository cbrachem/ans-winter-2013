%%%%%%%%%%%%%%%%%%%%%%%%%%%%%%%%%%%%%%%%%%%%%%%%%%%%%%%%%%%%%%%%%%%%%
%
%  This is a sample LaTeX input file for your contribution to 
%  the MC2013 conference. Modified by R.C. Martineau at INL from A. 
%  Sood at LANL, from J. Wagner ORNL who obtained the original class 
%  file by Jim Warsa, LANL, 16 July 2002}
%
%  Please use it as a template for your full paper 
%    Accompanying/related file(s) include: 
%       1. Document class/format file: mc2013.cls
%       2. Sample Postscript Figure:   figure.eps
%       3. A PDF file showing the desired appearance: template.pdf 
%    Direct questions about these files to: richard.martinea@inl.gov
%
%    Notes: 
%      (1) You can use the "dvips" utility to convert .dvi 
%          files to PostScript.  Then, use either Acrobat 
%          Distiller or "ps2pdf" to convert to PDF format. 
%      (2) Different versions of LaTeX have been observed to 
%          shift the page down, causing improper margins.
%          If this occurs, adjust the "topmargin" value in the
%          mc2013.cls file to achieve the proper margins. 
%
%%%%%%%%%%%%%%%%%%%%%%%%%%%%%%%%%%%%%%%%%%%%%%%%%%%%%%%%%%%%%%%%%%%%%


%%%%%%%%%%%%%%%%%%%%%%%%%%%%%%%%%%%%%%%%%%%%%%%%%%%%%%%%%%%%%%%%%%%%%
\documentclass{ansconf}
%
%  various packages that you may wish to activate for usage 
\usepackage{graphicx}
\usepackage{tabls}
\usepackage{hyperref}
%

%General Short-Cut Commands
\newcommand{\superscript}[1]{\ensuremath{^{\textrm{#1}}}}
\newcommand{\subscript}[1]{\ensuremath{_{\textrm{#1}}}}
%\newcommand{\nuc}[2]{\superscript{#2}{#1}}
\newcommand{\nuc}[2]{{#1}-{#2}}
\newcommand{\ith}[0]{$i^{\mbox{th}}$ }
\newcommand{\jth}[0]{$j^{\mbox{th}}$ }
\newcommand{\kth}[0]{$k^{\mbox{th}}$ }
\newcommand{\us}[0]{$\mu$s }

% New definition of square root:
% it renames \sqrt as \oldsqrt
\let\oldsqrt\sqrt
% it defines the new \sqrt in terms of the old one
\def\sqrt{\mathpalette\DHLhksqrt}
\def\DHLhksqrt#1#2{%
\setbox0=\hbox{$#1\oldsqrt{#2\,}$}\dimen0=\ht0
\advance\dimen0-0.2\ht0
\setbox2=\hbox{\vrule height\ht0 depth -\dimen0}%
{\box0\lower0.4pt\box2}}



% Insert authors' names and short version of title in lines below
%
\authorHead{Anthony M. Scopatz}
\shortTitle{PyNE Progress Report}

\confTitle{PyNE Progress Report}
\confLocation{Washington D.C., Nov. 11-14, 2013}
\confPublished{on CD-ROM (2013)}

%%%%%%%%%%%%%%%%%%%%%%%%%%%%%%%%%%%%%%%%%%%%%%%%%%%%%%%%%%%%%%%%%%%%%
%
%   BEGIN DOCUMENT
%
%%%%%%%%%%%%%%%%%%%%%%%%%%%%%%%%%%%%%%%%%%%%%%%%%%%%%%%%%%%%%%%%%%%%%
\begin{document}

\title{PyNE Progress Report}

\author{Anthony Scopatz\superscript{A}, John Xia\superscript{C}}
\affil{\textbf{A:} The University of Wisconsin-Madison, 
       \textbf{C:} The University of Chicago\\
       scopatz@wisc.edu}

\maketitle

\begin{abstract}
\raggedright

\emph{Key Words}: PyNE, MCNP, ENDF
\end{abstract}

\setlength{\baselineskip}{12pt}


\section{Enhanced Capabilities}

\subsection{Material Handling}

Improvements to material handling have been added to the \texttt{pyne.material}
and \texttt{pyne.mcnp} modules. A new attribute was added to the
\texttt{pyne.material.Material} class in order to store material metadata (e.g.
material name, the source of the material definition, comments).
\texttt{Material} class methods were added to write material definitions in MCNP
and ALARA \cite{wilson_alara:_1999} input file formats.  These materials
definitions are prepended by comments containing the aforementioned metadata. 

In addition, a function was added to \texttt{pyne.mcnp} to read MCNP input files and
create \texttt{Material} objects for each material specified in the file. This
function can parse formatted comments in MCNP input files in order to
populate the metadata fields of each \texttt{Material} object.

Finally, a new class, \texttt{pyne.material.MultiMaterial}, was added.  This
class stores collections of materials with along with corresponding mass or
volume fraction.  Class methods allow for mixing of these collections by mass
or volume fraction in order to create \texttt{ Material} objects.


\subsection{MCNP Support}

Two classes have been added to the \texttt{pyne.mcnp} module in order to store
MCNP data on Mesh Oriented datABase (MOAB) meshes \cite{tautges_moab:_2004}.
Storing data on MOAB meshes has numerous benefits including access to preexisting
mesh and data manipulation capabilities and easy visualization options. These
meshes are created using PyTAPS \cite{pytaps}, the Python interface to MOAB.

The \texttt{pyne.mcnp.Wwinp} class stores all of the information contained in a
single Cartesian MCNP WWINP file which contains a set of weight window lower
bounds for neutrons, photons, or both. The weight window lower bounds are
stored on a structured MOAB mesh attribute. In addition to class methods that
read-in and write-out MCNP WWINP files, \texttt{Wwinp} objects can be created
directly from MOAB meshes tagged with weight window lower bounds. This will
facilitate the future implementation of weight window generation schemes (e.g.
MAGIC \cite{davis_comparison_2011}, CADIS \cite{haghighat_monte_2003}) within
PyNE.

The \texttt{pyne.mcnp.Meshtal} class stores all the information from a Cartesian
MCNP MESHTAL file. These files may contain multiple neutron and/or photon mesh
tallies.  Tally results and relative errors are stored on structured MOAB
meshes. A dictionary attribute allows key-value access to these meshes using
tally numbers (i.e. 4, 14, 24, ...) as keys. Other methods allow for for the
addition and subtraction of tally values for different mesh tallies.

Another addition to PyNE's MCNP capabilities is the ability to read and
process particle track (PTRAC) files. The \texttt{PTRAC} card allows MCNP
users to record complete particle histories for selected events. However,
the custom binary file format used to save the particle tracks is cumbersome
to work with and varies in details, depending on system architecture or
how MCNP was compiled. The available plaintext format for particle
tracks is unfeasible due to the large size of the generated files and the
poor speed associated with processing plaintext files. Therefore, only
binary PTRAC files are supported by PyNE at this moment.

The \texttt{pyne.mcnp.PtracReader} class takes care of all the details
involved in reading particle track files and offers a simple interface
for reading event information and metadata associated with the run.
It also has a built-in function to write the particle tracks directly
to a file in the well-known HDF5 format \cite{hdf5}.

\subsection{ENDF Reader Support}

\section{Major Refactorizations}



\bibliographystyle{ans}
\bibliography{refs}

\end{document}
